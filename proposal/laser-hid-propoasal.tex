\documentclass[10pt]{article}

\usepackage[colorlinks=true]{hyperref}
\usepackage{setspace}
\usepackage[margin=1in]{geometry}
\usepackage{listings}
\usepackage{titlesec}

\usepackage{graphicx}
\graphicspath{ {images/} }
\usepackage{wrapfig}

\usepackage{indentfirst}
\usepackage{url}

\usepackage{amsmath}

\usepackage{fancyref}

% put all figures at the end of the document
%\usepackage[nomarkers,figuresonly]{endfloat}

% todo notes
\usepackage[colorinlistoftodos,prependcaption,textsize=tiny]{todonotes}

\usepackage{enumerate}
\usepackage{comment}

\newcommand{\ttinline}[1]{\texttt{\lstinline{#1}}}

\definecolor{mygreen}{rgb}{0,0.6,0}
\definecolor{mygray}{rgb}{0.5,0.5,0.5}
\definecolor{mymauve}{rgb}{0.58,0,0.82}

\lstset{ %
  backgroundcolor=\color{white},   % choose the background color
  basicstyle=\footnotesize,        % size of fonts used for the code
  breaklines=true,                 % automatic line breaking only at whitespace
  captionpos=b,                    % sets the caption-position to bottom
  commentstyle=\color{mygreen},    % comment style
  escapeinside={\%*}{*)},          % if you want to add LaTeX within your code
  %keywordstyle=\color{blue},       % keyword style
  %stringstyle=\color{mymauve},     % string literal style
}
\lstset{language=Python}

\begin{document}

\setlength{\fboxrule}{.5mm}\setlength{\fboxsep}{1.2mm}
\newlength{\boxlength}\setlength{\boxlength}{\textwidth}
\addtolength{\boxlength}{-4mm}
\begin{center}\framebox{\parbox{\boxlength}{\bf
ECE4760 \hfill Final Project Proposal \\
Istvan Burbank (ipb7), Peter Friend (pcf38), James Cassell (jcc384) \hfill
\today}}
\end{center}

% \singlespacing
% \titlespacing\section{0pt}{2pt plus 2pt minus 2pt}{2pt plus 2pt minus 2pt}
\titlespacing\subsection{0pt}{2pt plus 2pt minus 2pt}{2pt plus 2pt minus 2pt}
\titlespacing\subsubsection{0pt}{2pt plus 2pt minus 2pt}{2pt plus 2pt minus 2pt}

\section{Intro}
% A page describing the project at a high level (e.g. rationale, user interface, functions)

\section{Design}

\section{Standards}
% A paragraph describing the standards (IEEE, ISO, ANSI, etc) relevant to your project.

\section{Parts List}
% A tentative parts cost list.

% BEGIN RECEIVE ORGTBL component-table
\begin{tabular}{lll}
\hline
Component & Price & Link\\
\hline
 &  & \\
 &  & \\
\end{tabular}
% END RECEIVE ORGTBL component-table
\begin{comment}
  #+ORGTBL: SEND component-table orgtbl-to-latex
  |-----------+-------+------|
  | Component | Price | Link |
  |-----------+-------+------|
  |           |       |      |
  |           |       |      |
  % $ (optional extra dollar to keep font-lock happy, see footnote)
\end{comment}

\section{Impact}
% A paragraph describing the societal impact of your project. This might include:
\subsection{safety considerations}
\subsection{human factors and usability}
\subsection{interface design considerations for people with special needs}
\subsection{Intellectual property considerations}

\section{Hardware}
% A first pass of the program you will write, including:
% A block diagram or pseuocode
% Tentative source code (if you have it)
% A tentative schematic for any hardware you need to build.
% A parts list with specific parts types, so that we can see if we have the material. 

Our program will be broken into the following modules:
\begin{itemize}
\item Pixel Output
\item Frame Buffer
\item Scan Line Write
\item Input Module
\item Game Logic
\end{itemize}

The carousel will be wired through a switch to the power supply.
The lasers will be wired through FETs to the power supply.
The optical system will be wired just as in lab 4.

\end{document}


%%% Local Variables:
%%% mode: latex
%%% TeX-master: t
%%% End:
