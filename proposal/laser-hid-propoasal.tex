\documentclass[10pt]{article}

\usepackage[colorlinks=true]{hyperref}
\usepackage{setspace}
\usepackage[margin=1in]{geometry}
\usepackage{listings}
\usepackage{titlesec}

\usepackage{graphicx}
\graphicspath{ {images/} }
\usepackage{wrapfig}

\usepackage{indentfirst}
\usepackage{url}

\usepackage{amsmath}

\usepackage{fancyref}

% put all figures at the end of the document
%\usepackage[nomarkers,figuresonly]{endfloat}

% todo notes
\usepackage[colorinlistoftodos,prependcaption,textsize=tiny]{todonotes}

\usepackage{enumerate}
\usepackage{comment}

\newcommand{\ttinline}[1]{\texttt{\lstinline{#1}}}

\definecolor{mygreen}{rgb}{0,0.6,0}
\definecolor{mygray}{rgb}{0.5,0.5,0.5}
\definecolor{mymauve}{rgb}{0.58,0,0.82}

\lstset{ %
  backgroundcolor=\color{white},   % choose the background color
  basicstyle=\footnotesize,        % size of fonts used for the code
  breaklines=true,                 % automatic line breaking only at whitespace
  captionpos=b,                    % sets the caption-position to bottom
  commentstyle=\color{mygreen},    % comment style
  escapeinside={\%*}{*)},          % if you want to add LaTeX within your code
  %keywordstyle=\color{blue},       % keyword style
  %stringstyle=\color{mymauve},     % string literal style
}
\lstset{language=Python}

\begin{document}

\setlength{\fboxrule}{.5mm}\setlength{\fboxsep}{1.2mm}
\newlength{\boxlength}\setlength{\boxlength}{\textwidth}
\addtolength{\boxlength}{-4mm}
\begin{center}\framebox{\parbox{\boxlength}{\bf
ECE4760 \hfill Final Project Proposal \\
Istvan Burbank (ipb7), Peter Friend (pcf38), James Cassell (jcc384) \hfill
\today}}
\end{center}

% \singlespacing
% \titlespacing\section{0pt}{2pt plus 2pt minus 2pt}{2pt plus 2pt minus 2pt}
\titlespacing\subsection{0pt}{2pt plus 2pt minus 2pt}{2pt plus 2pt minus 2pt}
\titlespacing\subsubsection{0pt}{2pt plus 2pt minus 2pt}{2pt plus 2pt minus 2pt}

\section{Intro}
% A page describing the project at a high level (e.g. rationale, user interface, functions)
We propose to build a multi-color laser projector.
Unlike some laser-show projectors, our will be raster-based.
Some solutions exist like this one: \url{http://heim.ifi.uio.no/haakoh/avr/}, but we will use a more complex mechanism and more colors to achieve better resolution and a better picture.
We will also develop a game like Pacman or Snake to run on the console, controlled by a joystick.
Time permitting, we will also develop a more complex user input mechanism involving recording the projection surface.

\section{Design}
We will use two optical stages, one which spins (like in the video linked on the page above) and one which tilts.
The spinning element causes the lasers to scan left-to-right, while the tilting element select a scan line.
When outputting an image we will scan out the top line, then the next line, etc until we wrap around after the bottom line.
Three lasers will be used, red, green, and purple (blue is harder to find).
We don't expect to modulate their intensities, but any permutation of the colors is allowed in each of the pixels.

In more detail: the spinning stage will consist of several mirrors in a ``carousel'' mounted on top of a spinning platform, perhaps a PC cooling fan.
The carousel will spin at a fixed rate, but does not need to be regulated by the microcontroller.
The carousel will be outfitted with a photo resistor and LED similar to the setup we are using for lab 4 which will trigger the start of each scan line.
DMA will be used to output the pixels to the lasers, triggered by a clock.
Thus the screen width is determined by the clock rate AND the rotational speed of the carousel.

The vertical scanning is achieved by actuating a mirror with a variable-force-solenoid and spring combination from a speaker system.
Each time the photo resistor is triggered the mirror will move to the next scan line, wrapping to the top when appropriate.

The lasers will be toggled on and off with FETs.
The FETs are controlled by the DMA channels which will toggle the digital outputs.

The lasers will be calibrated to all be on the same scan line, but will be out of phase on that line.
By delaying the lasers by the appropriate number of degrees they can be brought back into phase.

\section{Standards}
% A paragraph describing the standards (IEEE, ISO, ANSI, etc) relevant to your project.

\section{Parts List}
% A tentative parts cost list.

% BEGIN RECEIVE ORGTBL component-table
\begin{tabular}{lll}
\hline
Component & Price & Link\\
\hline
 &  & \\
 &  & \\
\end{tabular}
% END RECEIVE ORGTBL component-table
\begin{comment}
  #+ORGTBL: SEND component-table orgtbl-to-latex
  |-----------+-------+------|
  | Component | Price | Link |
  |-----------+-------+------|
  |           |       |      |
  |           |       |      |
  % $ (optional extra dollar to keep font-lock happy, see footnote)
\end{comment}

\section{Impact}
% A paragraph describing the societal impact of your project. This might include:
\subsection{safety considerations}
\subsection{human factors and usability}
\subsection{interface design considerations for people with special needs}
\subsection{Intellectual property considerations}

\section{Hardware}
% A first pass of the program you will write, including:
% A block diagram or pseuocode
% Tentative source code (if you have it)
% A tentative schematic for any hardware you need to build.
% A parts list with specific parts types, so that we can see if we have the material. 

Our program will be broken into the following modules:
\begin{itemize}
\item Pixel Output
\item Frame Buffer
\item Scan Line Write
\item Input Module
\item Game Logic
\end{itemize}

The carousel will be wired through a switch to the power supply.
The lasers will be wired through FETs to the power supply.
The optical system will be wired just as in lab 4.

\end{document}


%%% Local Variables:
%%% mode: latex
%%% TeX-master: t
%%% End:
